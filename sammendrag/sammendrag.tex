\section*{Sammendrag}

Katalytisk partiell oksidasjon av metan (CPO) blir sett p� som en lovende m�te � produsere syntesegass fra metan. Reaksjonen er eksoterm og forholdsvis enkel � starte opp og avslutte.\\
I denne masteroppgaven har en katalysator best�ende av kobolt p� en \ce{CeO2-Al2O3}-b�rer, som har blitt kalsinert ved 1173 K-1473 K, blitt karakterisert gjennom \ce{N2}-adsorpsjon-desorpsjon, r�ntngendiffraksjon (XRD), temperaturprogrammert reduksjon (TPR) og \ce{H2}-kjemisorpsjon. Aktivitetstestingen har blitt utf�rt ved tre ulike ovnstemperaturer (923 K, 1023 K og 1123 K) med en GHSV p� 75 L$_\ce{CH4}$/g$_{\text{kat}} \cdot$t med et forhold mellom \ce{CH4},\ce{O2}og \ce{N2} p� henholdsvis 2, 1 og 3.72.\\
Form�let med denne oppgaven har v�rt � unders�ke den katalytiske aktiviteten til Co/\ce{CeO2-Al2O3} i katalytisk partiell oksidasjon av metan ved moderate temperaturer, og om den varierende konsentrasjonen av oksygenledighet i \ce{CeO2}-gitteret p�virker den katalytiske aktiviteten.

BET-overflatearealet til \ce{CeO2-Al2O3} sank ved �kende kalsineringstemperatur. \ce{CeO2} har vist seg � beskytte \ce{Al2O3} mot faseendringer ved h�ye temperaturer, noe som ble bekreftet ved sammenligning med rent \ce{Al2O3} kalsinert ved tilsvarende temperaturer som \ce{CeO2-Al2O3}. 
Krystallst�rrelsen og gitterparameteren til \ce{CeO2} ble funnet ved XRD. Krystallst�rrelsen �kte med �kende kalsineringstemperatur, mens det ikke ble funnet noen klar korrelasjon mellom gitterparameteren, krystallst�rrelsen og tilstedev�relsen av kobolt.\\
Dispersjonen til kobolt ble estimert ved \ce{H2}-kjemisorpsjon. Det ble funnet at dispersjonen for kobolt p� \ce{CeO2-Al2O3} kalsinert ved 1173 K, 1273 K og 1373 K var like. Dispersjonen for de �vrige katalysatorene �kte med b�rerens kalsineringstemperatur. Det er grunn til � tro at kjemisorpsjonsresultatene er p�virket av hydrogen "spillover" p� \ce{CeO2}, noe som betyr at en del av hydrogenmengden er adsorbert av \ce{CeO2}. Dette begrenser gyldigheten til resultatene.

Katalytisk partiell oksidasjon av metan over \textit{in situ}-reduserte katalysatorer ga h�y omsetning av metan og CO- og \ce{H2}-selektivitet. Co/\ce{CeO2-Al2O3} (1173 K), \\
(1273 K) and (1373 K) viste den beste yteevnen. En �kt aktivitet med minkende \ce{CeO2}-st�rrelse ble ikke observert. Resultatene underbygger teorien om en indirekte reaksjonsmekanisme, hvor fullstendig forbrenning foreg�r �verst i katalysatorlaget, mens endoterme reformeringsreaksjoner skjer nedstr�ms i katalysatorlaget.\\
Stabiliteten til katalysatorene ser ut til � v�re begrenset, sannsynligvis p� grunn av langsom oksidering av kobolt og/eller sintring. Det formodes at massetransport begrenser reaksjonen.\\
En cordierite monolitt, p�f�rt en \ce{Al2O3}-washcoat og impregnert med \ce{CeO2} og kobolt, ble ogs� testet i CPO ved 1023 K med en GHSV p� 8000 t$^{-1}$. Metanomsetningen og selektiviteten til CO og \ce{H2} var relativt h�y, men katalysatorens aktivitet sank rimelig raskt.
